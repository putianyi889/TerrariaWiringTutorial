\chapter{\texttt{TerrariaWiringHelper.jl}中文使用文档}

\texttt{TerrariaWiringHelper.jl}是一个辅助搭建电路的Julia工具箱。Julia是一种编程语言。

\section{Julia常见问题}
\paragraph{如何安装}
按照\myhref{https://julialang.org/install/}{Julia官网}的指示下载及安装Julia。

\paragraph{基本用法}
直接打开Julia程序,或者命令行输入\lstinline{julia},即可进入Julia终端(\autoref{fig88})。和Python类似,在Julia终端可以输入表达式输出结果,在命令行输入\lstinline{julia script.jl}可以执行相应的Julia脚本。
\begin{figure}[!htp]
    \centering
    \includegraphics[width=0.8\textwidth]{figure/juliaconsole.png}
    \caption{Julia终端}\label{fig88}
\end{figure}

\paragraph{VSCode支持}
如果使用VSCode IDE,可以安装\myhref{https://github.com/julia-vscode/julia-vscode}{Julia插件}。这个插件提供了直接执行代码、Workspace监控等方便的功能。

\paragraph{中文文档}
目前的Julia最新版本是1.12.3。\myhref{https://docs.juliacn.com/latest/}{中文文档}只支持到1.10.10,不过对于使用工具箱也够用了。

\paragraph{包管理模式和帮助模式}
在Julia终端中输入反方括号\lstinline{]}进入包管理模式(\autoref{fig90}),退格键删掉该方括号即可退出包管理模式。类似地,输入问号\lstinline{?}可以进入帮助模式(\autoref{fig91}),删除该问号退出帮助模式。帮助模式中输入某个函数名可以查看该函数的帮助文档。

\begin{figure}[!htp]
    \centering
    \subfloat[默认模式]{
        \includegraphics[height=\baselineskip]{figure/julianormal.png}\label{fig89}
    }
    \qquad
    \subfloat[包管理模式]{
        \includegraphics[height=\baselineskip]{figure/juliapkg.png}\label{fig90}
    }
    \qquad
    \subfloat[帮助模式]{
        \includegraphics[height=\baselineskip]{figure/juliahelp.png}\label{fig91}
    }
    \caption{Julia终端的三种模式}
\end{figure}

\section{工具箱的安装及使用}
\begin{lstlisting}[language=Julia, style=julia]
pkg> add https://github.com/putianyi889/TerrariaWiringHelper.jl # 直接从链接安装

julia> using TerrariaWiringHelper # 使用工具箱
    
pkg> update # 自动更新所有的包
    
pkg> remove TerrariaWiringHelper # 卸载
\end{lstlisting}

\section{$\mathbb{Z}_2$相关功能}
\begin{lstlisting}[language=Julia, style=julia]
julia> using TerrariaWiringHelper

julia> x = Z2Number(3) # 奇数视作1
1

julia> y = Z2Number(-2) # 偶数视作0
0

julia> x + x # Z_2中的加法
0
\end{lstlisting}

\subsection{$\mathbb{Z}_2$向量}
\begin{lstlisting}[language=Julia, style=julia]
julia> v = Z2Vector([1,0,1,1,0]) # 创建一个Z_2向量
5-element Z2Algebra.Z2RowVector{Vector{Z2Algebra.Z2Block}}:
1
0
1
1
0

julia> u = Z2Vector([0,1,1,0,1]); # 分号结尾隐藏输出

julia> AND(v) # 取与
0

julia> XOR(v) # 取异或
0

julia> u + v # 向量相加
5-element Z2Algebra.Z2RowVector{Vector{Z2Algebra.Z2Block}}:
1
1
0
1
1

julia> u .* v # 逐个元素相乘
5-element Z2Algebra.Z2RowVector{Vector{Z2Algebra.Z2Block}}:
0
0
1
0
0
\end{lstlisting}

\subsection{$\mathbb{Z}_2$矩阵}
\begin{lstlisting}[language=Julia, style=julia]
julia> A = Z2Matrix([1 0 0;1 1 0;1 1 1]) # 创建一个矩阵
3×3 Z2Matrix{Matrix{Z2Algebra.Z2Block}}:
1  0  0
1  1  0
1  1  1

julia> v = Z2Vector([0,1,1])
3-element Z2Algebra.Z2RowVector{Vector{Z2Algebra.Z2Block}}:
0
1
1

julia> A * v # 矩阵和向量相乘
3-element Z2Algebra.Z2RowVector{Vector{Z2Algebra.Z2Block}}:
0
1
0

julia> A \ v # 相除
3-element Z2Algebra.Z2RowVector{Vector{Z2Algebra.Z2Block}}:
0
1
0
\end{lstlisting}
