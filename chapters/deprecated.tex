\chapter{时代的眼泪}
本附录目的在于提醒读者在浏览古老的视频/帖子时注意当时的版本。

\section{液体复制}
在1.4版本之前,液体分流时会由于频繁的舍入误差明显增多或减少,所以通过一些让液体分流的装置就可以实现液体复制。1.4版本修改了液体流动的机制,使得这种方法无法用来复制液体了。

\section{烟花神教}
TODO

\section{旧版像素盒}

在1.4版本之前,如果像素盒上有横向电线激活且无纵向电线激活,那么像素盒熄灭;如果像素盒上既有横向电线激活又有纵向电线激活,那么像素盒点亮;如果无横向电线激活,那么像素盒不响应。同时,与一般的光源在亮灭之间切换不同,像素盒响应总是调整到对应状态。

使用这个版本的像素盒可以实现宽至多为24的密集矩阵显示器。因为其电路复杂,这里不给出电路图,仅给出原理。

首先来看如何更新屏幕状态。根据像素盒特性,显然每个像素盒都需要两个方向各一条线来控制,其中横向的线激活才会导致像素盒响应,仅纵向激活是不会响应的,这样一来屏幕的每行是独立的,可以逐行更新。

然后来看如何更新一行。当一行中的横向电线激活时,需要点亮的像素盒上的纵向电线必须被同一个电源激活,这意味着每个横向电线只能控制至多3个像素盒(\autoref{i258:259})。

\begin{figure}[!ht]
\begin{center}
\subfloat{
\label{i258}
\includegraphics{images/258.png}
}
\qquad
\subfloat{
\label{i259}
\includegraphics{images/259.png}
}
\end{center}
\caption{八个开关分别将像素盒状态变为000,001,010,011,100,101,110,111。}
\label{i258:259}
\end{figure}

幸亏像素盒本身就有分线盒的作用,这样一来每行可以安排八个横向电线,更新这一行时从中间向两侧,每三个为一组更新(\autoref{i260})。

\begin{figure}[!ht]
\centering
\includegraphics{images/260.png}
\caption{更新一行时,左边12个像素盒以3个为一组从右向左更新,右边12个像素盒以3个为一组从左向右更新。}
\label{i260}
\end{figure}

\section{金币掉落}
在1.4版本之前,当金币、沙块这一类需要支撑的物品掉落到无法生成图格的地方(如铁轨上)会被直接摧毁,所以要让这些物品变成掉落物需要一些技巧。