\chapter{时代的眼泪}
本附录目的在于提醒读者在浏览古老的视频/帖子时注意当时的版本。

\section{刷水机}
在1.4版本之前,液体分流时会由于频繁的舍入误差明显增多或减少,所以通过一些让液体分流的装置就可以实现液体复制。1.4版本修改了液体流动的机制,使得这种方法无法用来复制液体了。

尽管液体的流动机制是源码确定的,但是由于游戏环境下涉及到的液体格数过多,模型过于复杂,没有一个完善的理论来简单预测液体的行为。一般来说,分流会导致液体增加,而过度分流会导致液体减少。刷液体机大多数都是利用实体块分流液体。由于液体的机制尚不明确,刷液体机的分流构造也都是凭经验。

首先我们用物块搭一个有分流装置的小池子,并在底部放上入水泵,顶部放上出水泵,设置一个1秒计时器(\autoref{i7})。往池里倒上足量的水,然后用电线连接入水泵、出水泵和1秒计时器(\autoref{i8}),右键打开1秒计时器,那么每隔一秒,入水泵上的水被传送到出水泵,刷水机就开始工作了。可以虚化一格池壁来取水。由于不同液体流速不同,刷熔岩建议使用3秒计时器,刷蜂蜜建议使用5秒计时器。

这种构造的刷液体机有一个小缺点,那就是需要手动打开计时器,因为计时器在退出地图时会自动关闭。另一方面,水充满池子的时候,水泵无效,但计时器仍在运行,这对于强迫症来说是一个打击。有一个方法可以解决这个问题,那就是在水泵边上设置液体感应器(\autoref{i9},\autoref{i10})。如果水位够高,液体感应器保持为亮,水泵不工作。如果取水使得水位下降低于液体感应器,那么液体感应器熄灭,激活水泵,水泵抽水,抽出的水经过液体感应器时点亮液体感应器,液体感应器又激活水泵,直到水位达到液体感应器时液体感应器才因保持为亮而不激活电路。这个装置不仅做到了智能刷水,而且液体感应器这个驱动频率可以完美适配液体流速。

\begin{figure}[!ht]
	\begin{center}
		\subfloat[]{
			\label{i7}
			\includegraphics{images/7.png}
		}
		\subfloat[]{
			\label{i8}
			\includegraphics{images/8.png}
		}
		\subfloat[]{
			\label{i9}
			\includegraphics{images/9.png}
		}
		\subfloat[]{
			\label{i10}
			\includegraphics{images/10.png}
		}
	\end{center}
	\caption{}
	\label{i7:10}
\end{figure}

利用刷水机可以实现能摆在地图任意位置的、无延迟的开服感应器。打开地图的等待界面中有一项是“正在摆放液体”。这一步是将地图中所有不稳定的液体转移到最低处。如\autoref{i11:12},刷水机不断地生成水,水进入到左边的细长通道,并被下面的熔岩轨道吸收。当通道足够长并且刷水速度足够快时,通道中始终有水存在。此时退出地图并重新加载地图时,通道中的水被自动放置到最低处的液体感应器上,从而液体感应器激活。其余部分的功能是:将液体感应器上的水排走;打开刷水的1秒计时器。这个装置的触发是没有延迟的,但是会在游戏中一直运行刷水机,可能影响电脑性能。

\begin{figure}[!ht]
	\begin{center}
		\subfloat{
			\label{i11}
			\includegraphics[width=0.4\textwidth]{images/11.png}
		}
		\subfloat{
			\label{i12}
			\includegraphics[width=0.4\textwidth]{images/12.png}
		}
	\end{center}
	\caption{}
	\label{i11:12}
\end{figure}

\section{烟花神教}
TODO

\section{旧版像素盒}

在1.4版本之前,如果像素盒上有横向电线激活且无纵向电线激活,那么像素盒熄灭;如果像素盒上既有横向电线激活又有纵向电线激活,那么像素盒点亮;如果无横向电线激活,那么像素盒不响应。同时,与一般的光源在亮灭之间切换不同,像素盒响应总是调整到对应状态。

使用这个版本的像素盒可以实现宽至多为24的密集矩阵显示器。因为其电路复杂,这里不给出电路图,仅给出原理。

首先来看如何更新屏幕状态。根据像素盒特性,显然每个像素盒都需要两个方向各一条线来控制,其中横向的线激活才会导致像素盒响应,仅纵向激活是不会响应的,这样一来屏幕的每行是独立的,可以逐行更新。

然后来看如何更新一行。当一行中的横向电线激活时,需要点亮的像素盒上的纵向电线必须被同一个电源激活,这意味着每个横向电线只能控制至多3个像素盒(\autoref{i258:259})。

\begin{figure}[!ht]
\begin{center}
\subfloat{
\label{i258}
\includegraphics{images/258.png}
}
\qquad
\subfloat{
\label{i259}
\includegraphics{images/259.png}
}
\end{center}
\caption{八个开关分别将像素盒状态变为000,001,010,011,100,101,110,111。}
\label{i258:259}
\end{figure}

幸亏像素盒本身就有分线盒的作用,这样一来每行可以安排八个横向电线,更新这一行时从中间向两侧,每三个为一组更新(\autoref{i260})。

\begin{figure}[!ht]
\centering
\includegraphics{images/260.png}
\caption{更新一行时,左边12个像素盒以3个为一组从右向左更新,右边12个像素盒以3个为一组从左向右更新。}
\label{i260}
\end{figure}

\section{金币掉落}
在1.4版本之前,当金币、沙块这一类需要支撑的物品掉落到无法生成图格的地方(如铁轨上)会被直接摧毁,所以要让这些物品变成掉落物需要一些技巧。