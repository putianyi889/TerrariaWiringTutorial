\chapter{文档开发引导}

本文档使用\LaTeX 编写,在TeXLive 2025环境下编译。为了便于开发,文档导言区定义了一些命令与变量,本附录即为这些功能的文档。当然,可以直接在已有的代码中查看示例。

\section{三种排版主题}

本文档支持三种排版主题。优先保证电子版外观。

\begin{itemize}
  \item 打印版的文本内容最多。打印版不能用文本搜索功能,所以在末尾添加了索引;打印版不能点击超链接,所以链接均为明文。打印版的排版是针对双面打印的,页边距较宽,奇偶页边距不同。打印版通过编译\path{TerrariaWiringTutorial_printed.tex}生成。对于需要索引的关键词,用
  \item 电子版用于在电子屏幕上浏览。电子版的超链接地址全部隐藏用于节省空间,网站logo尽量采用小图标,页边距较窄,上下页边距极窄,便于电脑/手机浏览。电子版通过编译\path{TerrariaWiringTutorial_onscreen.tex}生成。
  \item 夜间模式是暗色主题的电子版,用于在低亮度环境下浏览。夜间模式通过编译\path{TerrariaWiringTutorial_nightmode.tex}生成。
\end{itemize}

\section{表格}
小型表格直接用\lstinline{tabular}环境。大型表格用\lstinline{longtable}宏包提供的同名环境。一个单元格内的换行,可以使用\lstinline{makecell}提供的同名命令。

\section{插图}

\subsection{游戏截图}
自制png/jpg插图放在\path{figures/}目录下。建议使用\lstinline{adjustbox}宏包的\lstinline{\adjincludegraphics}。如果是游戏内截图,建议用\wiki{夜明涂料}为图格打光,同时用\wiki{暗影漆}墙作为背景。截图建议保存为png格式并抠成透明背景。

\subsection{插图排版}
对于一行内多个插图,可以使用\lstinline|\smarthfill[每格大小,默认8pt]{插图总宽度,单位为格}[插图数量,默认为2]|调节间距。这个命令可以根据不同的页面宽度调整间距,使得间距均匀且至少2pt,至多2em。

\subsection{指令绘制}
可以用TikZ宏包直接绘制插图。关于TikZ的资料,互联网上非常丰富,AI也略知一二。这里介绍我们专门为\termTR 绘图创建的命令
\begin{itemize}
  \item \lstinline|\placetile{图格名称}{x}{y}|:以\lstinline{(x,y)}为左下角绘制一个大小为$1\times 1$的图格。图格名称即为\path{textures/tile/}目录下的文件名(不包含后缀)。例如\lstinline|\placetile{Torch_On}{0}{0}|。
  \item \lstinline|\placetiles{图格名称}{x1/y1,x2/y2,...}|:在很多位置绘制同一个图格。例如\lstinline|\placetiles{Torch_Off}{1/1,2/2,3/3}|。
  \item \lstinline|\placetileGroupN{图格名称}{x}{y}|:绘制尺寸大于$1\times 1$的图格。\lstinline{N}为宽度,\lstinline{(x,y)}为左下角。例如\lstinline|\placetileGroup2{Chest}{0}{0}|。
  \item \lstinline|\placeAndGate[灯数]{灯的状态}{门x}{门y}|:绘制一个与门。灯数是可选参数,若不提供则由灯的状态推断。灯的状态是01串,从上到下。例如\lstinline|\placeAndGate{110}{1}{2}|。
  \item \lstinline|\placeFaultyGate{灯的状态}{门x}{门y}|:绘制一个故障逻辑门。灯的状态中用2表示故障逻辑灯,其余同\lstinline{\placeAndGate}。
  \item \lstinline|\placeSevenSegment{七段线状态}|:以\lstinline{(0,0)}为左下角绘制一个火把七段线。每根线长度为2。七段线状态是01串,顺序是顶-左上-右上-中-左下-右下-底。例如显示4:\lstinline|\placeSevenSegment{0111010}|。
  \item \lstinline|\placePressurePlate{类型}{x}{y}|:绘制一个普通压力板。例如红压力板:\lstinline|\placePressurePlate{Red}{0}{0}|。
  \item \lstinline|\placeWeightedPressurePlate{类型}{状态}{x}{y}|:绘制一个测重压力板。例如弹起的粉色测重压力板:\lstinline|\placeWeightedPressurePlate{Pink}{Off}{0}{0}|。
  \item \lstinline|\playerSensorFrame{x}{y}|绘制一个位置在\lstinline{(x,y)}的玩家感应器的感应框。
\end{itemize}

以及模板
\begin{itemize}
  \item \lstinline{TRsch}:用于\lstinline{tikzpicture}的参数,将坐标和贴图对齐。
  \item \lstinline{solid tile}:画一片实体块。例如\lstinline{\draw[solid tile] (0,0) rectangle (3,2);}。
  \item \lstinline{water}:画一片水。例如\lstinline{\draw[water] (0,0) rectangle (3,2);}。
\end{itemize}

\subsection{示例}

\begin{lstlisting}[style=latex]
\begin{tikzpicture}[TRsch]
  \placetile{Torch_On}{0}{0}
  \placetile{Torch_Off}{2}{1}
  \draw[red] (0,0) -- (2,1);
\end{tikzpicture}%
\smarthfill{16}[4]%
\begin{tikzpicture}[TRsch]
  \placetiles{Switch_Flat_Off}{0/0,1/0,1/1}
  \draw[blue] (0,0) rectangle (2,2);
\end{tikzpicture}%
\smarthfill{16}[4]%
\begin{tikzpicture}[TRsch]
  \placetileGroup2{Gold_Chest}{0}{0}
  \placetileGroup3{Extractinator}{2}{1}
  \draw (0,0) grid[step=1] (5,4);
\end{tikzpicture}%
\smarthfill{16}[4]%
\begin{tikzpicture}[TRsch]
  \placeAndGate{1101}{0}{0}
  \placeAndGate{1111}{1}{0}
  \begin{scope}[xshift=32pt,yshift=-48pt]
    \placeSevenSegment{0111010}
  \end{scope}
\end{tikzpicture}
\end{lstlisting}
\begin{center}
  \begin{tikzpictureC}[TRsch]
    \placetile{Torch_On}{0}{0}
    \placetile{Torch_Off}{2}{1}
    \draw[red] (0,0) -- (2,1);
  \end{tikzpictureC}%
  \smarthfill{16}[4]%
  \begin{tikzpictureC}[TRsch]
    \placetiles{Switch_Flat_Off}{0/0,1/0,1/1}
    \draw[blue] (0,0) rectangle (2,2);
  \end{tikzpictureC}%
  \smarthfill{16}[4]%
  \begin{tikzpictureC}[TRsch]
    \placetileGroup2{Gold_Chest}{0}{0}
    \placetileGroup3{Extractinator}{2}{1}
    \draw (0,0) grid[step=1] (5,4);
  \end{tikzpictureC}%
  \smarthfill{16}[4]%
  \begin{tikzpictureC}[TRsch]
    \placeAndGate{1101}{0}{0}
    \placeAndGate{1111}{1}{0}
    \begin{scope}[xshift=32pt,yshift=-48pt]
      \placeSevenSegment{0111010}
    \end{scope}
  \end{tikzpictureC}
\end{center}

\section{链接}
\begin{itemize}
  \item \lstinline|\wiki[页面的实际名称,默认为显示名称]{显示名称}|:到\termTR 中文维基的链接
  \item \lstinline|\bilibili[#1,默认为video]{#2}|:生成链接\url{https://www.bilibili.com/#1/#2}
  \item \lstinline|\youtube{#1}|:生成链接\url{https://www.youtu.be/#1}
  \item \lstinline|\trforum[#1,默认为threads]{#2}|:生成链接\url{https://forums.terraria.org/index.php?#1/#2}
  \item \lstinline|\tieba{#1}|:生成链接\url{https://tieba.baidu.com/p/#1}
  \item \lstinline|\bbstr[#1,默认为t]{#2}|:生成链接\url{https://www.bbstr.net/#1/#2}
  \item \lstinline|\github{#1}|:生成链接\url{https://github.com/#1}
  \item \lstinline|\youku{#1}|:生成链接\url{https://v.youku.com/v_show/id_#1}
\end{itemize}

\section{预设颜色}

预设颜色定义在\path{preambles/skin.tex}中。

\begin{longtable}{|c|c|}
\hline
名称                                             & 描述 \\\hline
\endhead
\textcolor{bilibiliblue}{bilibiliblue}           & 哔哩哔哩蓝 \\\hline
\textcolor{youkudeeppink}{youkudeeppink}         & 优酷logo最深的粉色 \\\hline
\textcolor{terrariaforumback}{terrariaforumback} & 泰拉瑞亚官方论坛背景 \\\hline
\textcolor{terrariaforumtext}{terrariaforumtext} & 泰拉瑞亚官方论坛文本 \\\hline
\textcolor{bbstrback}{bbstrback}                 & 泰拉瑞亚中文论坛背景 \\\hline
\textcolor{bbstrtext}{bbstrtext}                 & 泰拉瑞亚中文论坛文本 \\\hline
\makecell{
  \textcolor{redwiredarkborder}{redwiredarkborder} \\
  \textcolor{redwirelightborder}{redwirelightborder} \\
  \textcolor{redwirelight}{redwirelight} \\
  \textcolor{redwiremiddle}{redwiremiddle} \\
  \textcolor{redwiredark}{redwiredark}
}                                                & 红色电线配色 \\\hline
\makecell{
  \textcolor{bluewiredarkborder}{bluewiredarkborder} \\
  \textcolor{bluewirelightborder}{bluewirelightborder} \\
  \textcolor{bluewirelight}{bluewirelight} \\
  \textcolor{bluewiremiddle}{bluewiremiddle} \\
  \textcolor{bluewiredark}{bluewiredark}
}                                                & 蓝色电线配色 \\\hline
\makecell{
  \textcolor{greenwiredarkborder}{greenwiredarkborder} \\
  \textcolor{greenwirelightborder}{greenwirelightborder} \\
  \textcolor{greenwirelight}{greenwirelight} \\
  \textcolor{greenwiremiddle}{greenwiremiddle} \\
  \textcolor{greenwiredark}{greenwiredark}
}                                                & 绿色电线配色 \\\hline
\makecell{
  \textcolor{yellowwiredarkborder}{yellowwiredarkborder} \\
  \textcolor{yellowwirelightborder}{yellowwirelightborder} \\
  \textcolor{yellowwirelight}{yellowwirelight} \\
  \textcolor{yellowwiremiddle}{yellowwiremiddle} \\
  \textcolor{yellowwiredark}{yellowwiredark}
}                                                & 黄色电线配色 \\\hline
\textcolor{thmcolback}{thmcolback}               & 定理类环境背景 \\\hline
\end{longtable}
