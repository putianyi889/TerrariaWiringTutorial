\chapter{代数理论}

目前代数理论可以用于解决组合逻辑与推广递次的电路压缩问题。本章内容面向有线性代数基础的读者。没有线性代数基础的读者需要预先学习向量、矩阵、线性方程组、行列式、秩的运算。群论基础对于本章内容的理解有帮助但是没有必要。

\section{域\texorpdfstring{$\mathbb{Z}_2$}{Z\_2}简介}
把整数分成奇数和偶数,则有如下运算规律:
\begin{center}
  \begin{tabular}{|c|cc|}
    \hline
    $+$ & 偶 & 奇 \\\hline
    偶  & 偶 & 奇 \\
    奇  & 奇 & 偶 \\\hline
  \end{tabular}
  \begin{tabular}{|c|cc|}
    \hline
    $\times$ & 偶 & 奇 \\\hline
    偶       & 偶 & 偶 \\
    奇       & 偶 & 奇 \\\hline
  \end{tabular}
\end{center}

$\mathbb{Z}_2=\{0,1\}$表示整数除以2所得余数的集合,则0代表偶数,1代表奇数,0和1之间的加法和乘法运算遵守上述奇偶的运算规律,即
\begin{center}
  \begin{tabular}{|c|cc|}
    \hline
    $+$ & 0 & 1 \\\hline
    0   & 0 & 1 \\
    1   & 1 & 0 \\\hline
  \end{tabular}
  \begin{tabular}{|c|cc|}
    \hline
    $\times$ & 0 & 1 \\\hline
    0        & 0 & 0 \\
    1        & 0 & 1 \\\hline
  \end{tabular}
\end{center}

另外,定义除法为乘法的逆运算,即$0\div 1=0$、$1\div 1=1$,除数不能为0。定义减法为加法的逆运算,由于$\mathbb{Z}_2$中恰好有$1=-1$,减法和加法的运算规则完全一样。

加法和乘法满足交换律和结合律,乘法对加法满足分配律。

$\mathbb{Z}_2$上的乘法和加法可以很好地描述\termTR 电路逻辑,如\autoref{fig83}所示。把多根线接到同一个输出上,相当于做加法;与逻辑相当于多个输入做乘法。

\begin{figure}[!htp]
  \centering
  \subfloat[]{
    \label{i350:351:2}
    \adjincludegraphics{images/350.png}%
    \quad%
    \adjincludegraphics{images/351.png}
  }%
  \subfloat[]{
    \label{i352:353:2}
    \adjincludegraphics{images/352.png}%
    \quad%
    \adjincludegraphics{images/353.png}
  }%
  \caption{
    \protect\subref{i350:351:2}加法:$B=A+C$;
    \protect\subref{i352:353:2}乘法:$C=AB$。
  }\label{fig83}
\end{figure}

\section{\texorpdfstring{$\mathbb{Z}_2$}{Z\_2}上的线性代数}
通过定义$\mathbb{Z}_2$上的加减乘除就可以导出$\mathbb{Z}_2$上的线性代数,基本理论也和实/复数域上线性代数没有什么区别。唯一需要注意的是,$\mathbb{Z}_2$上的多项式环结构不同于实/复数域,所以相应的特征值理论有区别,这个区别将在后续小节中细说。本小节将侧重于解释为什么我们需要用到$\mathbb{Z}_2$上的线性代数。

$\mathbb{Z}_2$上的向量和矩阵在电路中有实际意义,因为接线的本质就是矩阵。

\subsection{接线的矩阵描述}
\begin{example}{}{}
  一个七段线显示器的状态可以表示成一个$\mathbb{Z}_2$上的7维向量。考虑如图所示的七段线显示结构,七个输入并不直接对应七段线。输入的状态表示为$\mathbf{x}=(x_1,\dots,x_7)$,七段线状态表示为$\mathbf{y}=(y_1,\dots,y_7)$。

  \begin{center}
    \begin{tikzpicture}[scale=0.5]
      \draw[rounded corners, fill] (0,0) rectangle node[white] {$y_7$} (3,1);
      \draw[rounded corners, fill] (0,1) rectangle node[white] {$y_5$} (-1,4);
      \draw[rounded corners, fill] (3,1) rectangle node[white] {$y_6$} (4,4);
      \draw[rounded corners, fill] (0,4) rectangle node[white] {$y_4$} (3,5);
      \draw[rounded corners, fill] (0,5) rectangle node[white] {$y_2$} (-1,8);
      \draw[rounded corners, fill] (3,5) rectangle node[white] {$y_3$} (4,8);
      \draw[rounded corners, fill] (0,8) rectangle node[white] {$y_1$} (3,9);

      \draw[red, ultra thick] (-0.25,3.75) -- (-0.25,0.75) -- (3.25,0.75) -- (3.25,3.75) -- (10,3.75) node[thmcoltext, anchor=west] {$x_5$};
      \draw[cyan, ultra thick] (0.25,0.5) -- (3.5,0.5) -- (3.5,3.5) -- (3.5,3.0) -- (10,3.0) node[thmcoltext, anchor=west] {$x_6$};
      \draw[green, ultra thick] (3.75,1.25) -- (3.75,3.5) -- (3.75,2.25) -- (10,2.25) node[thmcoltext, anchor=west] {$x_7$};

      \draw[red, ultra thick] (-0.25,5.25) -- (-0.25,8.25) -- (3.25,8.25) -- (3.25,5.25) -- (10,5.25) node[thmcoltext, anchor=west] {$x_3$};
      \draw[cyan, ultra thick] (0.25,8.5) -- (3.5,8.5) -- (3.5,5.5) -- (3.5,6) -- (10,6) node[thmcoltext, anchor=west] {$x_2$};
      \draw[green, ultra thick] (3.75,7.75) -- (3.75,5.5) -- (3.75,6.75) -- (10,6.75) node[thmcoltext, anchor=west] {$x_1$};

      \draw[brown, ultra thick] (-0.5,1.25) -- (-0.5,7.75);
      \draw[brown, ultra thick] (-0.5,4.5) -- (10,4.5) node[thmcoltext, anchor=west] {$x_4$};
    \end{tikzpicture}
  \end{center}

  七个输入和七段线的对应关系可以用一个矩阵表示:

  \[
    \begin{array}{c|ccccccc}
      % \begin{noindent}
          & y_1 & y_2 & y_3 & y_4 & y_5 & y_6 & y_7 \\\hline
      x_1 &     &     &  1  &     &     &     &     \\
      x_2 &  1  &     &  1  &     &     &     &     \\
      x_3 &  1  &  1  &  1  &     &     &     &     \\
      x_4 &     &  1  &     &  1  &  1  &     &     \\
      x_5 &     &     &     &     &  1  &  1  &  1  \\
      x_6 &     &     &     &     &     &  1  &  1  \\
      x_7 &     &     &     &     &     &  1  &     \\\hline
      % \end{noindent}
    \end{array}
  \]

  %表格中每行表示一个输入覆盖的段,可以理解为一个7维行向量;每列表示经过一个分段的输入,可以理解为一个7维列向量。去掉表头,这个表格就是一个$\mathbb{Z}_2$上的$7\times 7$\myind{矩阵},记为$\mathbf{A}$。

  %一个分段的值等于经过它的输入的和,例如$y_3=x_1+x_2+x_3$。这可以写成$y_3=1x_1+1x_2+1x_3+0x_4+0x_5+0x_6+0x_7$,即$y_3$是输入$\mathbf{x}$和矩阵中对应列向量的\myind{向量积}:

  %\[
  %  y_3=\mathbf{x}(1,1,1,0,0,0,0)^\top,
  %\]
  记这个矩阵为$\mathbf{A}^\top$,七段线的初始状态为$\mathbf{b}$,则分段与输入的关系可以表示为$\mathbf{y}=\mathbf{A}\mathbf{x}+\mathbf{b}$.

\end{example}

\subsection{组合逻辑}
任何一个$n$输入的组合逻辑都可以看作是一个从$\mathbb{Z}_2^n$到$\{0,1\}$的函数。这个函数对于某些输入取值为1,其他输入取值为0。对于这个函数的性质的研究,可以转化为对于那些取值为1的输入的研究。
\begin{definition}{特征集}{}
  一个组合逻辑的输入数称为该逻辑的\emph{维数}。设某$n$维组合逻辑可以用$n$元函数$f$表示,使函数值为1的输入的集合写作$f^{-1}(1)=\{\mathbf{v}\in\mathbb{Z}_2^n:f(\mathbf{v})=1\}$,这个集合称为$f$的\emph{特征集}。特征集的秩称为该逻辑的\emph{秩}。$n$维组合逻辑和$\mathbb{Z}_2^n$的子集存在一一对应关系。
\end{definition}
\begin{example}{}{}
  与逻辑的特征集有且仅有一个元素。$n$维异或逻辑的特征集是$\mathbb{Z}_2^n$的标准基。
\end{example}
\begin{theorem}{TNoName \& putianyi888}{}
  假设某$n$维组合逻辑可以用单异或门实现,则该异或门的最优灯数不超过$n+1$。
\end{theorem}
\begin{proof}
  用$V=\mathbb{Z}_2^n$表示输入空间,$U=\mathbb{Z}_2^m$表示灯的状态空间。用矩阵$\boldsymbol{M}\in\mathbb{Z}_2^{m\times n}$表示接线,向量$\boldsymbol{b}\in U$表示灯的初始状态,用$L:V\to U, \boldsymbol{v}\mapsto \boldsymbol{M}\boldsymbol{v}+\boldsymbol{b}$表示从输入到灯的状态的映射。设某$n$维逻辑为$f:V\to\mathbb{Z}_2$,实现该逻辑的异或门为$g:U\to \mathbb{Z}_2^n$。假设$m>n+1$。

  我们的目标是找到$U$的一组基$\boldsymbol{F}=(\boldsymbol{B}|\boldsymbol{C}|\boldsymbol{A})$,使得
  \begin{enumerate}
    \item $L(f^{-1}(1))=\boldsymbol{B}$;
    \item $\boldsymbol{C}\cap L(V)=\emptyset$;
    \item $L(V) \subset \spanspace(\boldsymbol{B})\oplus\spanspace(\boldsymbol{C})$,其中后者的维数$d\le n+1$。
  \end{enumerate}
  这样一来,记$F^{-1}:\boldsymbol{v}\mapsto\boldsymbol{F}^{-1}\boldsymbol{v}$是将$\boldsymbol{F}$映到标准基的线性映射。考虑异或门上新的接线$F^{-1}L$,便有
  \begin{enumerate}
    \item $F^{-1}L(f^{-1}(1))$是$\abs{\boldsymbol{B}}$个标准基向量;
    \item $F^{-1}L(f^{-1}(0))$不含标准基向量;
    \item $F^{-1}L(V)$除了前$d$个分量以外全部为0,所以可以去掉这些恒为0的灯,留下前$d$个灯。
  \end{enumerate}
  综上,接线$F^{-1}L$和$L$表示的逻辑等价,且在$F^{-1}L$中可以只保留$d\le n+1$个灯。下面我们对于齐次和非齐次两种情况分别构造$\boldsymbol{F}$。

  先考虑齐次情况$\boldsymbol{b}=\boldsymbol{0}$。由于$\dim(V)=n$,$L(V)$是$U$的一个$n'\le n$维线性子空间。由于$g$是异或逻辑,$g^{-1}(1)$是$U$的标准基,且$L(f^{-1}(1))=g^{-1}(1)\cap L(V)$。记$L(f^{-1}(1))$为$\boldsymbol{B}$,将其扩充为$L(V)$的一组基$(\boldsymbol{B}|\boldsymbol{C})$,再扩充为$U$的一组基$(\boldsymbol{B}|\boldsymbol{C}|\boldsymbol{\alpha}_1|\boldsymbol{A})$,其中$\boldsymbol{A}=(\boldsymbol{\alpha}_2,\dots,\boldsymbol{\alpha}_{m-n'})$(可能为空)。$U$的另一组基$\boldsymbol{F}=(\boldsymbol{B}|\boldsymbol{C}+\boldsymbol{\alpha}_1|\boldsymbol{\alpha}_1|\boldsymbol{A})$可以满足要求,其中$\boldsymbol{C}+\boldsymbol{\alpha}_1$表示将$\boldsymbol{\alpha}_1$加到$\boldsymbol{C}$的每一列上。

  对于非齐次情况$\boldsymbol{b}\ne\boldsymbol{0}$,如果$\boldsymbol{b}\in \boldsymbol{M}V$,那么$L(V)=\boldsymbol{M}V$,退化为齐次情况。如果$L(V)\cap g^{-1}(1)=\emptyset$,则$f(V)=g(L(V))=\{0\}$,退化为平凡逻辑。

  一般地,记$\boldsymbol{B}=L(V)\cap g^{-1}(1)=(\boldsymbol{b}_1,\dots,\boldsymbol{b}_{n'})$,其中$n'\ge 1$,那么$L(V)=\boldsymbol{M}V+\boldsymbol{b}_1$。记$\boldsymbol{B}'=\boldsymbol{B}\backslash\{\boldsymbol{b}_1\}$(可能为空),则$\boldsymbol{B}'-\boldsymbol{b}_1$是$\boldsymbol{M}V$中的线性无关组,将其扩充为$\boldsymbol{M}V$的一组基$(\boldsymbol{B}'-\boldsymbol{b}_1|\boldsymbol{C})$。注意到$(\boldsymbol{B}|\boldsymbol{C})$是$\spanspace(L(V))=\boldsymbol{M}V\oplus\{\boldsymbol{b_1}\}$的一组基,将其扩充为$U$的一组基$\boldsymbol{F}=(\boldsymbol{B}|\boldsymbol{C}|\boldsymbol{A})$,则$\boldsymbol{F}$满足要求。

\end{proof}

\subsection{推广递次}

\section{\texorpdfstring{$\mathbb{Z}_2$}{Z\_2}上的多项式}
