\chapter{零散机制}
本附录用于收录一些不成系统的机制。

\section{物品掉落速度}
\begin{note}
来源:\url{https://forums.terraria.org/index.php?threads/item-fall-speed.76050/}
\end{note}
\begin{itemize}
\item 真空中的物品,重力加速度为1像素/帧$^2$,最大掉落速度为7像素/帧。by:pbq
\item 熔岩和水中的物品,重力加速度为0.08像素/帧$^2$,最大掉落速度为5像素/帧;蜂蜜中的物品,重力加速度为0.05像素/帧$^2$,最大掉落速度为3像素/帧。by:DRKV,通过\lstinline{Terraria.Item.UpdateItem(int)}
\item 新生成的掉落物拥有的初始垂直速度大约在4像素/帧到1.5像素/帧。by:pbq
\end{itemize}

\section{雕像怪移动方式}
\begin{note}
来源:\url{https://forums.terraria.org/index.php?threads/project-engineering-cheat-sheet-statue-mob-characteristics.75881/}

最后更新:2015.1.3;版本:1.2.4.1
\end{note}
\begin{itemize}
\item 在夜晚,所有雕像怪都会在生成后向人物移动。\wiki{鸟}(会飞离人物)和\wiki{宝箱怪}(会保持宝箱形态)是例外。在白天,\wiki{骷髅}会向右移动。
\item 在夜晚,敌怪会追踪人物。\wiki{兔兔}、\wiki{金鱼}和\wiki{鸟}会保持一个方向运动,直到遇到障碍物。
\item \wiki{鲨鱼}、\wiki{水母}和\wiki{食人鱼}只有在被液体淹没时才能移动。它们只会追踪淹没在2格及以上液体中的人物。
\item \wiki{鸟}在被液体淹没时会追踪人物。
\item \wiki{史莱姆}在液体中会上浮。
\item \wiki{金鱼}在旱地上会弹跳。
\item 只有\wiki{骷髅}、\wiki{螃蟹}和\wiki{蝙蝠}可以下平台。这几种NPC在白天会自动下平台,而在夜晚只有当人物在平台之下时才会下平台。在夜晚,这几种NPC还会在平台上弹跳。
\item 所有雕像怪和人物的最大掉落速度相同:大约37格/秒。有些雕像怪似乎加速度慢一些。蝙蝠可以以约6格/秒的速度向下飞。鸟只有在遇到斜面天花板的时候才会向下飞。
\item 雕像生成的金鱼和兔兔会遵循一般游戏机制,在雨天、血月、被使用魔粉后发生转化。
\item 转化后的NPC不占用雕像的生成限制。尽管如此,在同一帧中生成的金鱼或兔兔还是会被限制,可能因为它们要在下一次刷新才会发生转化。
\item 雕像怪不计入活跃敌怪数。
\end{itemize}