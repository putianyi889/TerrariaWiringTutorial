\chapter{冷却时间机制}\label{app35}
\begin{note*}{}{}
本机制属于1.3.5.3版本。
\end{note*}
\begin{table}[!ht]
\centering
\begin{tabular}{|c|c|}
\hline
图格名称&冷却时间/帧\\\hline\hline
炮台&30\\\hline
雪球发射器&10\\\hline
烟花盒&30\\\hline
烟花喷泉&30\\\hline
矿车轨道&5\\\hline
飞镖机关&200\\\hline
超级飞镖机关&200\\\hline
烈焰机关&200\\\hline
长矛机关&90\\\hline
尖球机关&300\\\hline
喷泉(机关)&200\\\hline
刷怪雕像&30\\\hline
刷物品雕像&600\\\hline
传送NPC雕像&300\\\hline
\end{tabular}
\caption{使用冷却时间机制的图格及其冷却时间。}\label{tab6}
\end{table}

有不少电路物品都有冷却时间(\autoref{tab6})。游戏使用一个大小为1000的列表(以下称为冷却列表)存储冷却中的图格及其剩余冷却时间。每帧中,游戏刷新整个冷却列表,把其中所有图格的剩余冷却时间减1,如果减到0则将该图格从冷却列表中删除。

此外,还有一些图格没有冷却时间,但是也借用了冷却列表用来做倒计时,包括计时器和引爆器。

将某图格加入冷却列表时,将其加入\textbf{从前向后}的第一个空位,如果该图格已经存在冷却列表中,或者冷却列表已满,则加入失败。除引爆器以外,所有其他冷却图格如果没能成功加入冷却列表,则不会响应激活。所以,不光图格正在冷却时无法激活,在整个冷却列表已满时也无法激活。

冷却列表的刷新顺序是\textbf{从后向前}。

将某图格从冷却列表中删除时,会将冷却列表中其后的所有图格依次向前移一个位置,使得冷却列表始终保持从前往后连续填充的特性。