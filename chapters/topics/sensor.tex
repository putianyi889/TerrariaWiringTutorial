\section{传感器}
泰拉瑞亚中已经自带了很多传感器,例如压力板、感应器等。在实际使用中,我们有时需要检测游戏自带传感器检测不了的东西,那么就需要另外设计传感器。

\subsection{开服感应器}
所谓开服感应器,就是在打开地图时激活的电源。只要地图不关闭,该电源就不会再次激活。

最容易想到的就是在重生点放上玩家感应器,并把玩家传送走。但是这样一来,在玩家回程时会再次激活。使用床传送技术\youku{XMTg0NzYxNDg0OA}可以避免这一点,但是无法在退出地图时自动重置。

从打开地图开始,玩家第一次接近傀儡时傀儡影子会在傀儡上生成。傀儡是家具而傀儡影子是敌怪,它们分开结算:家具是固定的,而影子可以移动、受伤害、受debuff。虽然影子不会主动移动,但是会被动移动,例如坠落、传送机传送、半砖传送。当影子受伤害时家具播放动画。影子与家具所在格由于各自的原因均不能摆放前景物块\footnote{前景物块不能摆放在家具图格上或生物碰撞箱上。}。

只要傀儡影子在重生点附近,那么傀儡影子在打开地图时就会重新生成在傀儡上。

最简单的开服感应器如\autoref{i221}。傀儡悬空,下面放有压力板。每次打开地图后傀儡影子生成,掉落在压力板上激活压力板。这个开服感应器略有延迟,延迟长度是傀儡影子掉落到压力板上的时间。一般来说这么短的延迟不会有什么问题。

\begin{figure}[!ht]
  \centering
  \adjincludegraphics{images/221.png}
  \caption{}
  \label{i221}
\end{figure}

如果想要更短的延迟,可以考虑在开图瞬间用传送机将傀儡影子传送走,并做一些处理使传送机再次激活时不会将影子传送回(\autoref{i222})。

\begin{figure}[!ht]
  \centering
  \adjincludegraphics{images/222.png}
  \caption{}
  \label{i222}
\end{figure}

上面的装置中,传送机激活后1帧,傀儡影子就被半砖推离传送机并触发红压力板,从而不会再传回。但是玩家感应器和测重压力板不会触发传送机,玩家出生在重生点处也不会触发普通压力板。只能用玩家感应器或测重压力板触发机关,然后用青绿压力垫板激活传送机,这样一来在触发机关到青绿压力垫板激活之间又有一个短延迟。

\subsection{方向感应器}
方向感应器被广泛地使用在电路游戏中,它可以检测到玩家的上下左右移动操作。方向感应器的一种方案如\autoref{i253:254}所示。另一种方案请参考\bilibili{av23633364/?p=7}。

\begin{figure}[!ht]
  \centering
  \adjincludegraphics{images/253.png}%
  \smarthfill{22}%
  \adjincludegraphics{images/254.png}
  \caption{玩家试图左右移动时会触发玩家感应器并实化半砖将玩家推回;试图上平台时平台虚化,玩家掉落;试图下平台时下方半砖实化将玩家推回。}
  \label{i253:254}
\end{figure}

\subsection{刷怪感应器}
刷怪感应器应用于刷怪场和Boss战场中,用来检测各种敌怪生成。大多数敌怪都可以直接触发压力板,所以一般情况下压力板就可以用来做刷怪感应器。穿墙怪不会触发压力板,只能使用特殊方法处理。

一个勉强及格的方案是在玩家周围放置压力板,穿墙怪攻击玩家时,玩家被击退,触发压力板,从而敌怪被检测到。这个方案的缺点是明显的,但是暂时也没有更好的方法。

\subsection{传送感应器}
\begin{wrapfigure}{R}{88pt}
  \centering
  \begin{tikzpictureC}[TRsch]
    \draw[solid tile] (0,0) rectangle (3,1);
    \placeWeightedPressurePlate{Pink}{Off}{1}{1}
    \placetile{Player_Sensor_Off}{1}{-2}
    \placeAndGate{01}{4}{-2}
    \placeFaultyGate{20}{6}{-2}
    \draw[red wire] (1,1) -- (4,1) -- (4,0);
    \draw[blue wire] (1,-2) -- (1,-1) -- (4,-1);
    \draw[green wire] (4,-2) -- (5,-2) -- (5,-1) -- (6,-1) -- (6,0);
    \draw[red wire] (6,-2) -- (8,-2);
    \drawPlayerSensorArea{1}{-2}
  \end{tikzpictureC}
  \caption{传送感应器,右下红线是输出。设计\&测试:光明牛奶、绿叶ly}
  \label{fig95}
\end{wrapfigure}
传送感应器可以感应玩家瞬移到某固定地点,例如出生点。传送感应器的原理是测重压力板的灵敏度大于玩家感应器,而玩家感应器的范围大于测重压力板。用玩家感应器和测重压力板检测同一个地点,如果玩家是移动进去,则先触发玩家感应器;如果玩家是瞬移进去,则先触发测重压力板。当然,如果玩家移速非常快,一帧以内就可以从玩家感应器的范围外移动到了测重压力板上,则传送感应器也会认为玩家是在“瞬移”。

\autoref{fig95}的传送感应器可以检测的瞬移包括传送机、\wiki{混沌传送杖}、\wiki{魔法海螺}、\wiki{恶魔海螺}、\wiki{贝壳电话}(出生点)。因为某种原因,不支持单人模式的重生点传送(即\wiki{魔镜}),但支持多人模式的。
