\section{\texttt{SceneMetrics.cs}}\label{src:SceneMetrics}

\subsection{\texttt{SceneMetrics.ScanOnScreenTiles}}\label{src:ScanOnScreenTiles}
这个函数主要用于检测类似于\wiki{天塔柱}这样的对附近一定范围有影响的图格。目前不确定这个范围是多少,但似乎是屏幕范围+上下左右各30格。在1.4.5版本之前,那些图格都是分开结算的,因此影响范围有区别,而1.4.5版本将它们都合并到了一个函数中,共享一个范围。受其影响的机制如下:

\begin{longtable}{|ccc|}
\hline
图格ID & 名称 & 影响变量 \\\hline
\endhead
\hline
\endfoot
104 & \wiki{落地大摆钟} & \texttt{SceneMetrics.HasClock} \\
139 & \wiki{八音盒} & \texttt{SceneMetrics.ActiveMusicBox} \\
207 & \wiki{喷泉} & \texttt{SceneMetrics.ActiveFountainColor} \\
410 & 四柱\wiki{天塔柱} & \texttt{SceneMetrics.ActiveMonolithType} \\
480 & \wiki{血月天塔柱} & \texttt{SceneMetrics.BloodMoonMonolith} \\
509 & \wiki{虚空天塔柱} & \texttt{SceneMetrics.ActiveMonolithType} \\
657 & \wiki{回声腔} & \texttt{SceneMetrics.EchoMonolith} \\
658 & \wiki{以太天塔柱} & \texttt{SceneMetrics.ShimmerMonolithState} \\
720 & \wiki{CRT天塔柱} & \texttt{SceneMetrics.CRTMonolith} \\
721 & \wiki{复古天塔柱} & \texttt{SceneMetrics.RetroMonolith} \\
725 & \wiki{电影放映机} & \texttt{SceneMetrics.NoirMonolith} \\
733 & \wiki{收音机} & \texttt{SceneMetrics.RadioThingMonolith} \\
\end{longtable}

扫描顺序是从左到右、从上到下、列主序。当范围内有多个同类图格时,只有最后一个生效。受其影响的有八音盒、喷泉和(四柱+虚空天塔柱)。其他机制有独立的flag控制。
