\chapter{刷怪机制}
执行刷怪的函数为 NPC.SpawnNPC。

如果NPC.noSpawnCycle为true,那么这一帧不刷怪,把NPC.noSpawnCycle重置为false。

如果要刷怪的话,会对每个未死亡的玩家执行刷怪过程。在史莱姆雨进行时,会在其他刷怪之前插入史莱姆雨的刷怪。史莱姆雨的刷怪是额外刷怪,不会影响正常刷怪。

进行正常刷怪时,首先要判断刷怪类型、刷怪率和刷怪量,然后决定是否要刷怪,然后决定刷怪点和刷怪面,最后决定刷什么怪。

\section{史莱姆雨刷怪}

\section{刷怪类型、刷怪率和刷怪量}

\section{刷怪点和刷怪面}

\section{刷怪种类}
刷怪的优先级是四柱>太空>事件>昏迷男子>蜘蛛洞>地下沙漠>巨骨舌鱼>血水母>嗜血怪>海洋刷怪>沉睡渔夫>食人鱼>蓝水母>水中小动物。

海洋刷怪中包含部分沉睡渔夫。琵琶鱼包含在食人鱼中。绿水母包含在蓝水母中。

\subsection{四柱刷怪}
如果玩家在四柱附近(玩家中心到四柱中心的直线距离不大于4000像素),那么就会刷四柱怪。四柱刷怪的优先级是星云柱>星旋柱>星尘柱>日曜柱。

\subsubsection{星云柱}
刷怪为星云浮怪(ID:420)、吮脑怪(ID:421)、进化兽(ID:423)、预言帝(ID:424)。这四个怪的刷怪比例为1:5:3:3。星云浮怪在整个世界中的上限为2个,进化兽在整个世界中的上限为3个,预言帝在整个世界中的上限为2个。这个刷怪比例\&上限的规则是四柱的特色。举例来说,没有任何刷怪的时候,这四个怪的刷怪概率分别是1/12、5/12、1/4、1/4;如果已经刷出了两个星云浮怪,那么不会再刷星云浮怪,剩下三个怪的刷怪概率分别是5/11、3/11、3/11。

\subsubsection{星旋柱}
刷怪为漩泥怪(ID:425)、异星蜂王(ID:426)、异星黄蜂(ID:427)、星旋怪(ID:429)。刷怪比例为2:1:2:4。漩泥怪上限为3,异星蜂王上限为3,星旋怪上限为4。

\subsubsection{星尘柱}
刷怪为银河织妖(ID:402)、星细胞(ID:405)、流体入侵怪(ID:407)、闪耀炮手(ID:409)、观星怪(ID:411)。刷怪比例为1:1:1:2:3。

\subsubsection{日曜柱}
刷怪为千足蜈蚣(ID:412)、火龙怪(ID:415)、火龙怪骑士(ID:416)、火滚怪(ID:417)、流星火怪(ID:418)、火月怪(ID:419)、火龙战士(ID:518)。刷怪比例为1:1:1:1:1:1:1。千足蜈蚣上限为1,火龙怪上限为2,火龙怪骑士上限为1,火龙战士上限为2。

\subsection{太空刷怪}
优先级是火星飞船>火星探测器>飞龙>鸟妖。

如果执行的是火星入侵的刷怪,那么生成火星飞船(ID:388)。

在石巨人后,并且刷怪点到世界中心的横坐标距离大于世界宽度$\times$0.165\footnote{小世界为693格,中世界为1056格,大世界为1386格},那么有概率生成火星探测器(ID:399),这个概率与是否打过火星入侵、是否在水蜡烛区域、是否有水蜡烛buff相关(\autoref{tab5651})。水蜡烛区域和水蜡烛buff不是一回事,水蜡烛区域不包括手持水蜡烛的情况。火星探测器的上限为1。

\begin{table}[!h]
    \centering
    \begin{tabular}{cccccc}
         000&001&011&100&101&111\\\hline
         1/8&15/64&5/9&1/30&59/900&19/100 
    \end{tabular}
    \caption{二进制的第一位表示是否打过火星入侵,第二位表示是否在水蜡烛区域,第三位表示是否有水蜡烛buff。例如101表示打过火星入侵,不在水蜡烛区域内,有水蜡烛buff。}
    \label{tab5651}
\end{table}

没有水蜡烛buff的时候,飞龙(ID:87)的生成概率为1/10;有水蜡烛buff的时候生成概率为19/100。飞龙的上限为1。当玩家中心在人工背景墙前时不会刷飞龙。

如果火星飞船、火星探测器、飞龙均未生成,那么生成鸟妖。

\subsection{事件刷怪}
\subsubsection{哥布林入侵}
1/9概率生成哥布林巫士(ID:29),8/45概率生成哥布林苦力(ID:26),32/135概率生成哥布林弓箭手(ID:111),32/405概率生成哥布林盗贼(ID:27),64/405概率生成哥布林战士(ID:28)。在困难模式中,有1/30概率生成哥布林召唤师(ID:471),这会覆盖前面的生成。哥布林召唤师上限为1。

\subsubsection{雪人入侵}
1/7概率生成巴拉雪人(ID:145),2/7概率生成雪人暴徒(ID:143);4/7概率生成戳刺先生(ID:144)。

\subsubsection{海盗入侵}
1/11概率生成海盗弩手(ID:215),10/99概率生成鹦鹉(ID:252),80/693概率生成海盗神射手(ID:214),160/693概率生成私船海盗(ID:213),320/693概率生成海盗水手(ID:212)。

海盗船长(ID:216)有1/30概率生成,会覆盖前面的生成。海盗船长上限为1。

海盗船(ID:491)生成要求入侵进度超过一半,并且刷怪面的左右各20格,上方10格到40格范围内没有实体块。海盗船生成概率是1/20。海盗船上限为1。海盗船的生成会覆盖其他生成。

\subsubsection{火星入侵}
火星飞碟(ID:395)的生成分为两段判定。离入侵结束不到100分\footnote{入侵事件总分为160+40$\times$玩家数量}时,火星飞碟的概率为1/10并且会覆盖其他生成(包括第二段)。第二段中火星飞碟和其他敌怪处理方法相同。

火星飞碟上限为1,火星走妖(ID:520)上限为1。以下是第二段判定。

火星飞碟概率为1/70,鳞甲怪枪手(ID:390)和火星工程师(ID:386)概率均为9/140(火星飞碟达到上限的话,这个概率变为1/14),火星飞船(ID:388)概率为2/35,扰脑怪(ID:381)概率为4/35,激光枪手(ID:382)概率为4/35,火星走妖(ID:520)概率为1/7,灰咕噜兽(ID:385)、电击怪(ID:389)和火星军官(ID:383)概率均为1/7(火星走妖达到上限的话,这个概率为4/21)。

\subsection{昏迷男子}
满足昏迷男子生成条件,并且不是水中刷怪,那么有1/80概率生成昏迷男子。

\subsection{蜘蛛洞}
刷怪面有蜘蛛墙,不高于 rockLayer,距离世界底端大于210格,且不是水中刷怪,未解救过发型师,那么有1/8概率生成受缚发型师(ID:354)。

在未生成受缚发型师的前提下,如果刷怪面有蜘蛛墙或者判定为蜘蛛洞刷怪,那么困难模式生成黑隐士(ID:163),困难模式前生成爬墙蜘蛛(ID:164)。

\subsection{地下沙漠刷怪}
进行地下沙漠刷怪的要求是刷怪面在地下,并且刷怪面上的背景墙是硬化沙墙/沙岩墙,或者它们的转化墙\footnote{神圣化、腐化、血腥化}。刷怪优先级是沙虫>墓穴爬虫>困难模式>其他。

沙虫(ID:510)和墓穴爬虫(ID:513)的生成都要求玩家中心不能在人工背景墙前,并且刷怪面在地表下100格以下。沙虫的概率为1/33,墓穴爬虫的概率为1/22。沙虫只会在困难模式生成。

在困难模式中,有4/5的概率进行困难模式刷怪。困难模式的刷怪有腐恶食尸鬼(ID:525)、红染食尸鬼(ID:526)、神梦食尸鬼(ID:527)、食尸鬼(ID:524)、沙漠幽魂(ID:533)、邪恶拉弥亚(ID:529)、沙贼(ID:530)、拉弥亚(ID:528)、蛇蜥怪(ID:532),它们的刷怪比例为2:2:2:2:1:1:1:1:1,腐恶食尸鬼在腐化环境生成,红染食尸鬼在血腥环境生成,神梦食尸鬼在神圣环境生成,食尸鬼只在纯净环境生成,沙漠幽魂和邪恶拉弥亚在腐化或血腥环境生成,沙贼和拉弥亚在没有腐化和血腥的环境生成,蛇蜥怪不挑环境。举例来说,如果玩家同时处在血腥和神圣环境中,那么可以生成红染食尸鬼、神梦食尸鬼、沙漠幽魂、邪恶拉弥亚、蛇蜥怪,其刷怪比例为2:2:1:1:1。

既没刷出沙虫或墓穴爬虫,也没有困难模式刷怪,那么刷蚁狮(ID:69)/蚁狮马(ID:508)/蚁狮蜂(ID:509),刷怪比例为1:3:1。

\subsection{巨骨舌鱼}
困难模式+水中刷怪+丛林环境,2/3概率生成巨骨舌鱼(ID:157)。

\subsection{血水母}
困难模式+水中刷怪+血腥环境,2/3概率生成血水母(ID:242)。

\subsection{嗜血怪}
困难模式+水中刷怪+血腥环境,2/3概率生成嗜血怪(ID:241)。

\subsection{海洋刷怪}
要求:水中刷怪,刷怪横坐标距离地图左右边界<250格,刷怪面是沙块或珍珠/黑檀/猩红沙块,刷怪面纵坐标在rockLayer之上。优先级:沉睡渔夫>其他。

\subsubsection{沉睡渔夫}
要求:刷怪横坐标在安全区域外,满足沉睡渔夫生成条件。

从刷怪面向上搜索47格(不包括刷怪面)找到第一个可以生成沉睡渔夫的图格,这一格需要满足:没有实体块;上方第一格没有实体块;上方第二格没有液体;上方第二格没有实体块。如果找到了这一格,那么生成沉睡渔夫(ID:376)。

\subsubsection{其他}
1/60概率生成海蜗牛(ID:220),59/1500概率生成乌贼(ID:221),59/500概率生成鲨鱼(ID:65),413/1500概率生成螃蟹(ID:67),413/750概率生成粉水母(ID:64)。

\subsection{沉睡渔夫}
要求:不是水中刷怪,刷怪横坐标距离地图左右边界<340格,刷怪面是沙块或珍珠/黑檀/猩红沙块刷怪面纵坐标在worldSurface之上,满足沉睡渔夫生成条件。生成沉睡渔夫。

\subsection{食人鱼}
要求:水中刷怪。刷怪面在rockLayer之下,有1/2概率生成食人鱼(ID:58);刷怪面是丛林草皮,必然生成食人鱼。在困难模式,生成的食人鱼有2/3概率转化为琵琶鱼(ID:102)。

\subsection{蓝水母}
要求:水中刷怪,刷怪面在worldSurface之下。有1/3概率生成蓝水母(ID:63)。困难模式中生成的蓝水母转化为绿水母(ID:103)。

\subsection{水中小动物}
如果是水中刷怪,有1/4概率生成腐化小动物。如果在腐化环境,生成腐化金鱼(ID:57)。如果不在腐化环境,刷怪面在worldSurface之上,刷怪面距离世界顶端大于50格,在白天,有1/6概率在水面\footnote{这里的水面判定与海洋沉睡渔夫的判定相同。}等概率生成鸭(ID:364)或野鸭(ID:362)。上一句话中没有成功生成的话,生成金鱼(ID:55)。

\subsection{受缚哥布林}
要求:满足受缚哥布林生成条件,不是水中刷怪,刷怪面不在rockLayer之上,刷怪面距离世界底端大于210格。有1/20概率生成受缚哥布林(ID:105)。

\subsection{受缚巫师}
要求:满足受缚巫师生成条件,,不是水中刷怪,刷怪面不在rockLayer之上,刷怪面距离世界底端大于210格。有1/20概率生成受缚巫师(ID:106)。


