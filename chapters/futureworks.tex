\chapter{待实现的电路或装置}

\section{电子钟}
原理很简单,使用假人驱动计数器。需要实现功能:精确到游戏秒,显示月相,进地图启动,使用日晷后会校正时间。可考虑的功能:12小时制与24小时制转化,自定义闹钟。

尽管有人已经做了一个电子钟\footnote{\url{https://www.bilibili.com/video/av55613747}},但是它没有实现12小时与24小时转化功能与闹钟功能,也没有精确到游戏秒。

\section{确定有限状态自动机}
虽然确定有限自动机的组合逻辑构造与状态转换表有关,但是自动机的整体结构还没有一个优秀的可用方案。主要的难点可能在于接线,因为自动机中的线路是循环的,这违背了泰拉瑞亚中逻辑电路的一般规律。

\section{俄罗斯方块}
像素盒显示器的宽度限制是非常讨厌的,这使它无法实现我们一般习惯的横向显示器。另一方面,由于宽度被限制在24格之内,其实际能显示的东西非常有限。俄罗斯方块是为数不多的像素盒显示器可以胜任的任务。经典的俄罗斯方块主显示屏宽10格,高20格。副显示屏用来显示下一个方块,2*4或4*2皆可。使用方向感应器控制。

有人已经做过俄罗斯方块了\footnote{\url{https://www.bilibili.com/video/av38924330}},但是它不能变速,不能计分,操作也不方便。

\section{贪吃蛇}
贪吃蛇是像素盒显示器能胜任的另一个任务。其缺点是像素盒显示器是单色的,可能无法分辨身体和果实。屠宰场与 TheRedStoneCrafter 已经分别做了一些工作。

\section{魔方}
使用彩线灯泡做显示。

有人已经做过魔方了\footnote{\url{https://www.bilibili.com/video/av56760618}},但是它操作极不方便。

\section{Flappy Bird}
虽然 TheRedStoneCrafter 已经做过了Flappy Bird\footnote{\url{https://www.bilibili.com/video/av24265449}},但是其效果并不好,还原度可以更高。首先是人物控制,可以利用水+海神贝壳模拟 Flappy Bird 中的运动,用尖刺或木尖刺模拟障碍物,检测人物位置,当人物被击退时游戏结束。

虽然使用递次电路可以实现,但是其占地过大,使得纵向上每个单元至少占三格。使用1秒计时器串联是最节省空间,静态视觉效果最好的,但是其频率过低。

可以用蜂蜜中的飞镖模拟障碍物,但是飞镖在蜂蜜中几乎不可见,可以使用弹幕踏板激活其他光源或虚实化尖刺来提高视觉效果。使用飞镖的好处就是电路难度较低,只需要设计发射飞镖的电路,其余的动画效果都可以自动完成。

\section{2048}
使用四向操纵板操作。因为多位数显面积太大,用指数表示,0表示2,1表示4,2表示8,……,9表示1024,A表示2048。如果要超过2048,用b表示4096,c表示8192,d表示16384,E表示32768,F表示65536,G表示131072(可能的最大值)。

使用像素盒显示可以显示出移动动画,但是难度相当大。

\section{计算机}