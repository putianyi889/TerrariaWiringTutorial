\usepackage{graphicx}
\usepackage{bitset}
\usepackage{etoolbox}
\usepackage{xstring}

\usetikzlibrary{arrows.meta}
\usetikzlibrary{calc}
\usetikzlibrary{decorations.pathreplacing}
\usetikzlibrary{calligraphy}
\usetikzlibrary{fill.image}

\newcommand{\SolidTile}{Smooth_Marble_Block}
\newcommand{\PlatformPlain}{Stone_Platform}
\newcommand{\PlatformLava}{Stone_Platform}

\tikzset{
  TRsch/.style = {
    scale=0.5,
    transform shape,
    x=16pt,
    y=16pt
  },
  water/.style = {opacity=0.2, blue},
  solid tile/.style = {
    preaction={
      clip,
      postaction={
        draw=black,
        line width=2pt,
        line join=bevel,
        fill tile image*={width=16pt}{\TexturePath\SolidTile_C.png}
      }
    },
    draw=none
  },
}

\newcommand{\addwirestyle}[1]{
  \expandafter\tikzset{%
    #1 wire/.style={
      preaction={
        draw,
        line cap=round,
        line width=2pt,
        #1wirelightborder,
        xshift=8pt,
        yshift=8pt
      },
      line cap=round,
      line width=1pt,
      #1wirelight,
      xshift=8pt,
      yshift=8pt
    }
  }
}
\addwirestyle{red}
\addwirestyle{blue}
\addwirestyle{green}
\addwirestyle{yellow}

% \begin{noindent}
\NewDocumentCommand{\drawRedBlue}{O{}mO{}m}{
  \begin{scope}[blend group=difference]
    \begin{scope}[blend group=normal]
      \draw[red wire,#1] #2
    \end{scope}
    \begin{scope}[blend group=normal]
      \draw[blue wire,#3] #4
    \end{scope}
  \end{scope}
}
% \end{noindent}

% \begin{noindent}
\NewDocumentCommand{\drawGreenYellow}{O{}mO{}m}{%
  \begin{scope}[blend group=soft light]
    \begin{scope}[blend group=hue]
      \begin{scope}[blend group=normal]
        \draw[yellow wire,#3] #4
      \end{scope}
      \begin{scope}[blend group=normal]
        \draw[green wire,#1] #2
      \end{scope}
    \end{scope}
    \begin{scope}[blend group=hue]
      \begin{scope}[blend group=normal]
        \draw[green wire,#1] #2
      \end{scope}
      \begin{scope}[blend group=normal]
        \draw[yellow wire,#3] #4
      \end{scope}
    \end{scope}
  \end{scope}
}
% \end{noindent}

\newcommand{\placetile}[3]{%
  \node at ($(#2,#3)+(0.5,0.5)$) {\includegraphics[width=16pt]{\TexturePath#1.png}};
}

\newcommand{\placetiles}[2]{%
  \foreach \x/\y in {#2}{
    \placetile{#1}{\x}{\y}
  }
}

% 第一个参数是宽度,第二个参数是文件名,第三第四个参数是左下格坐标,第四个参数是宽度
\newcommand{\placetileGroup}[4]{%
  \begingroup
  \pgfmathsetmacro{\w}{16*#1}
  \node[anchor=south west] at (16pt*#3-4pt,16pt*#4-4pt) {\includegraphics[width=\w pt]{\TexturePath#2.png}};
  \endgroup
}

\newcommand{\placeSevenSegment}[1]{
  \bitsetSetBin{sevenSeg}{#1}
  \bitsetQuery{sevenSeg}{6}{
    \placetile{Torch_On}{1}{6}
    \placetile{Torch_On}{2}{6}
  }{
    \placetile{Torch_Off}{1}{6}
    \placetile{Torch_Off}{2}{6}
  }
  \bitsetQuery{sevenSeg}{5}{
    \placetile{Torch_On}{0}{5}
    \placetile{Torch_On}{0}{4}
  }{
    \placetile{Torch_Off}{0}{5}
    \placetile{Torch_Off}{0}{4}
  }
  \bitsetQuery{sevenSeg}{4}{
    \placetile{Torch_On}{3}{5}
    \placetile{Torch_On}{3}{4}
  }{
    \placetile{Torch_Off}{3}{5}
    \placetile{Torch_Off}{3}{4}
  }
  \bitsetQuery{sevenSeg}{3}{
    \placetile{Torch_On}{1}{3}
    \placetile{Torch_On}{2}{3}
  }{
    \placetile{Torch_Off}{1}{3}
    \placetile{Torch_Off}{2}{3}
  }
  \bitsetQuery{sevenSeg}{2}{
    \placetile{Torch_On}{0}{2}
    \placetile{Torch_On}{0}{1}
  }{
    \placetile{Torch_Off}{0}{2}
    \placetile{Torch_Off}{0}{1}
  }
  \bitsetQuery{sevenSeg}{1}{
    \placetile{Torch_On}{3}{2}
    \placetile{Torch_On}{3}{1}
  }{
    \placetile{Torch_Off}{3}{2}
    \placetile{Torch_Off}{3}{1}
  }
  \bitsetQuery{sevenSeg}{0}{
    \placetile{Torch_On}{1}{0}
    \placetile{Torch_On}{2}{0}
  }{
    \placetile{Torch_Off}{1}{0}
    \placetile{Torch_Off}{2}{0}
  }
}

\newcommand{\placeAndGate}[4][]{
  % 第一个参数(可选):灯的数量;第二个参数是灯的状态,从上到下;后两个参数是门的坐标
  \begingroup
  \if\relax\detokenize{#1}\relax
    \StrLen{#2}[\mylength]%
  \else
    \edef\mylength{#1}%
  \fi
  \bitsetSetBin{bits}{#2}
  \pgfmathsetmacro\N{\mylength-1}
  \foreach \i in {0,...,\N} {
    \pgfmathsetmacro\y{\i+#4+1}
    \bitsetQuery{bits}{\i}{
      \placetile{Logic_Lamp_On}{#3}{\y}
    }{
      \placetile{Logic_Lamp_Off}{#3}{\y}
    }
  }
  \ifnumcomp{\bitsetCardinality{bits}}{=}{\mylength}{
    \placetile{And_Gate_On}{#3}{#4}
  }{
    \placetile{And_Gate_Off}{#3}{#4}
  }
  \endgroup
}

\newcommand{\placeXorGate}[4][]{
  % 第一个参数(可选):灯的数量;第二个参数是灯的状态,从上到下;后两个参数是门的坐标
  \begingroup
  \if\relax\detokenize{#1}\relax
    \StrLen{#2}[\mylength]%
  \else
    \edef\mylength{#1}%
  \fi
  \bitsetSetBin{bits}{#2}
  \pgfmathsetmacro\N{\mylength-1}
  \foreach \i in {0,...,\N} {
    \pgfmathsetmacro\y{\i+#4+1}
    \bitsetQuery{bits}{\i}{
      \placetile{Logic_Lamp_On}{#3}{\y}
    }{
      \placetile{Logic_Lamp_Off}{#3}{\y}
    }
  }
  \ifnumcomp{\bitsetCardinality{bits}}{=}{1}{
    \placetile{Xor_Gate_On}{#3}{#4}
  }{
    \placetile{Xor_Gate_Off}{#3}{#4}
  }
  \endgroup
}

\newcommand{\placeFaultyGate}[3]{
  % 第一个参数是灯的状态,从上到下,2表示故障灯;后两个参数是门的坐标
  \begingroup
  \bitsetSetOct{bits}{#1}
  \StrLen{#1}[\mylength]
  \pgfmathsetmacro\N{\mylength-1}
  \foreach \i in {0,...,\N} {
    \pgfmathsetmacro{\faulty}{int(3*\i+1)}
    \pgfmathsetmacro{\lamp}{int(3*\i)}
    \pgfmathsetmacro{\y}{\i+#3+1}
    \bitsetQuery{bits}{\faulty}{
      \placetile{Logic_Lamp_Faulty}{#2}{\y}
    }{
      \bitsetQuery{bits}{\lamp}{
        \placetile{Logic_Lamp_On}{#2}{\y}
      }{
        \placetile{Logic_Lamp_Off}{#2}{\y}
      }
    }
  }
  \placetile{Faulty_Gate}{#2}{#3}
  \endgroup
}

\newcommand{\placePressurePlate}[3]{
  \pgfmathsetmacro\y{#3-0.125}
  \placetile{#1_Pressure_Plate}{#2}{\y}
}
\newcommand{\placeWeightedPressurePlate}[4]{
  \pgfmathsetmacro\y{#4-0.25}
  \placetile{#1_Weighted_Pressure_Plate_#2}{#3}{\y}
}

\definecolor{playerSensorFrame}{RGB}{32, 178, 170}
\newcommand{\drawPlayerSensorArea}[2]{%
  \begingroup
  \pgfmathsetmacro{\l}{#1-2.0625}
  \pgfmathsetmacro{\r}{#1+3.0625}
  \pgfmathsetmacro{\b}{#2+0.9375}
  \pgfmathsetmacro{\t}{#2+11.0625}
  \draw[playerSensorFrame, line width=1pt] (\l,\b) rectangle (\r,\t);
  \endgroup%
}
