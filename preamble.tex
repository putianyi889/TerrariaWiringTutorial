\usepackage{amsmath}
\usepackage{amssymb}
\usepackage{graphicx}
\usepackage{graphbox}
\usepackage{float}
\usepackage{etoolbox}
%\tikzsetexternalprefix{sch/}

\if\rendermode1
	\usepackage[a4paper, left=2cm, right=2cm, top=0.5cm, bottom=0.5cm]{geometry}
\else\if\rendermode2
\else\if\rendermode3
	\usepackage[a4paper, left=2cm, right=2cm, top=0.5cm, bottom=0.5cm]{geometry}
\fi\fi\fi

\usepackage{adjustbox}
\adjustboxset{scale=0.666667, max width=\linewidth}

\usepackage[colorlinks]{hyperref}
\hypersetup{
	unicode=true,
	pdfencoding=unicode,
	bookmarksdepth=5,
}
\pdfstringdefDisableCommands{
	\def\myind#1{#1}
}

\if\rendermode1
	\NewDocumentCommand{\myind}{m}{#1}
\else
	\usepackage{imakeidx}
	\makeindex
	\NewDocumentCommand{\myind}{m}{#1\index{#1}}
\fi

\usepackage{seqsplit}
\newcommand{\trvar}[1]{\texttt{\seqsplit{#1}}}

\usepackage{unicode-math}

\usepackage{xunicode}
\newfontfamily\unisym{DejaVu Sans}

\usepackage{newunicodechar}

\newunicodechar{⬯}{{\setmathfont{XITS Math}$\mathord\whtvertoval$}}
\newunicodechar{✗}{{\unisym ✗}}
\newunicodechar{□}{$\square$}
% 配图、配色

\usepackage{xcolor}
\usepackage{xstring}

\definecolor{bilibiliblue}{RGB}{35,173,229}
\definecolor{terrariaforumback}{RGB}{47, 50, 81}
\definecolor{terrariaforumtext}{RGB}{247, 232, 209}
\definecolor{bbstrtext}{RGB}{35, 52, 91}
\definecolor{bbstrback}{RGB}{241, 250, 245}
\definecolor{youkudeeppink}{RGB}{255, 45, 100}

\definecolor{grassforest}{HTML}{1CD85E}
\definecolor{dirtblock}{HTML}{976B4B}
\definecolor{torch}{HTML}{FDDD03}

\definecolor{redwiredarkborder}{RGB}{59, 0, 1}
\definecolor{redwirelightborder}{RGB}{109, 13, 15}
\definecolor{redwirelight}{RGB}{253, 58, 61}
\definecolor{redwiremiddle}{RGB}{218, 2, 5}
\definecolor{redwiredark}{RGB}{172, 2, 5}

\definecolor{bluewiredarkborder}{RGB}{1, 35, 61}
\definecolor{bluewirelightborder}{RGB}{13, 67, 109}
\definecolor{bluewirelight}{RGB}{83, 180, 253}
\definecolor{bluewiremiddle}{RGB}{2, 124, 218}
\definecolor{bluewiredark}{RGB}{2, 99, 172}

\definecolor{greenwiredarkborder}{RGB}{1, 61, 26}
\definecolor{greenwirelightborder}{RGB}{13, 109, 53}
\definecolor{greenwirelight}{RGB}{83, 253, 153}
\definecolor{greenwiremiddle}{RGB}{2, 218, 91}
\definecolor{greenwiredark}{RGB}{2, 172, 72}
\usepackage{amsthm}
\usepackage{tcolorbox}
\tcbuselibrary{theorems}
\tcbuselibrary{breakable}
\tcbuselibrary{skins}

\tcbset{label separator={}}
\newcommand{\addwireframe}[1]{
  \expandafter\tcbset{%
    #1frame/.style={%
      borderline={1pt}{2pt}{#1wirelightborder},
      borderline={1pt}{-1pt}{#1wiredarkborder},
      borderline={1pt}{0pt}{#1wiredark, dash pattern= on 1pt off 1pt},
      borderline={1pt}{1pt}{#1wirelight, dash pattern= on 1pt off 1pt},
      borderline={1pt}{0pt}{#1wiremiddle, dash pattern= on 1pt off 1pt, dash phase=1pt},
      borderline={1pt}{1pt}{#1wiremiddle, dash pattern= on 1pt off 1pt, dash phase=1pt}
    }
  }
}
\addwireframe{blue}
\addwireframe{red}
\addwireframe{green}
\addwireframe{yellow}

\newtcbtheorem{theorem}{定理}{
  enhanced,
  sharp corners=all,
  colframe=greencolframe, colback=thmcolback, coltext=thmcoltext,
  fonttitle=\bfseries,
  greenframe
}{}
\newtcbtheorem{corollary}{推论}{
  enhanced,
  sharp corners=all,
  colframe=greencolframe, colback=thmcolback, coltext=thmcoltext,
  fonttitle=\bfseries,
  greenframe
}{}

\theoremstyle{definition}
\newtcbtheorem{example}{例}{
  breakable,
  colframe=bluecolframe, colback=thmcolback, coltext=thmcoltext,
  skin=enhanced,
  parbox=false,
  sharp corners=all,
  before title={\noindent},
  fonttitle=\bfseries,
  blueframe
}{}
\newtcbtheorem{definition}{定义}{
  breakable,
  colframe=greencolframe, colback=thmcolback, coltext=thmcoltext,
  skin=enhanced,
  parbox=false,
  sharp corners=all,
  before title={\noindent},
  fonttitle=\bfseries,
  greenframe
}{}
\newtcbtheorem{problem}{思考题}{
  breakable,
  colframe=yellowcolframe, colback=thmcolback, coltext=thmcoltext,
  skin=enhanced,
  parbox=false,
  sharp corners=all,
  before title={\noindent},
  fonttitle=\bfseries,
  yellowframe
}{}

\theoremstyle{remark}
\newtcbtheorem{note}{注意}{
  enhanced,
  sharp corners=all,
  colframe=redcolframe, colback=thmcolback, coltext=thmcoltext,
  fonttitle=\bfseries,
  redframe
}{}
\newtcbtheorem{remark}{注}{
  enhanced,
  sharp corners=all,
  colframe=redcolframe, colback=thmcolback, coltext=thmcoltext,
  fonttitle=\bfseries,
  redframe
}{}

\makeatletter
\newcommand{\tcb@cnt@exampleautorefname}{例}
\makeatother

\usepackage{multicol}
\setlength{\columnseprule}{1pt}
\ifcase\rendermode
\or
  \newenvironment{maybecols}{\begin{multicols}{2}}{\end{multicols}}
\or
  \newenvironment{maybecols}{}{}
\or
  \newenvironment{maybecols}{\begin{multicols}{2}}{\end{multicols}}
\fi

% 小型模板

\usepackage{tcolorbox}

\if\rendermode1
	\NewDocumentCommand{\myhref}{mm}{\href{#1}{#2}}
	
	\NewDocumentCommand{\bilibili}{O{video}m}{\href{https://www.bilibili.com/#1/#2}{\includegraphics[align=c, height=9pt]{figure/bilibiliicon.pdf}}}
	\newcommand{\youtube}[1]{\href{https://youtu.be/#1}{\includegraphics[align=c, height=9pt]{figure/youtubeicon.pdf}}}
	\NewDocumentCommand{\trforum}{O{threads}m}{\href{https://forums.terraria.org/index.php?/#1/#2}{\includegraphics[align=c, height=9pt]{figure/terrariaforum.png}}}
	\newcommand{\tieba}[1]{\href{https://tieba.baidu.com/p/#1}{\includegraphics[align=c, height=9pt]{figure/tieba.png}}}
	\NewDocumentCommand{\bbstr}{O{t}m}{\href{https://www.bbstr.net/#1/#2}{\includegraphics[align=c, height=9pt]{figure/bbstr.png}}}
	\NewDocumentCommand{\github}{m}{\href{https://github.com/#1}{\includegraphics[align=c, height=9pt]{figure/githubicon.pdf}}}
	\NewDocumentCommand{\youku}{m}{\href{https://v.youku.com/v_show/id_#1}{\includegraphics[align=c, height=9pt]{figure/youku.pdf}}}
\else
	\NewDocumentCommand{\myhref}{mm}{#2\index{#2}\footnote{\url{#1}}}
	
	\NewDocumentCommand{\bilibili}{O{video}m}{\href{https://www.bilibili.com/#1/#2}{\tcbox[on line, boxsep=1pt, colframe=bilibiliblue, colback=white, left=0pt, right=0pt, top=0pt, bottom=0pt]{\includegraphics[align=c, height=9pt]{figure/bilibili.pdf} \footnotesize\sffamily\textcolor{bilibiliblue}{#2}}}}
	\newcommand{\youtube}[1]{\href{https://youtu.be/#1}{\tcbox[on line, boxsep=1pt, colframe=black, colback=white, left=0pt, right=0pt, top=0pt, bottom=0pt]{\includegraphics[align=c, height=9pt]{figure/youtube.pdf} \footnotesize\sffamily\textcolor{black}{#1}}}}
	\NewDocumentCommand{\trforum}{O{threads}m}{\href{https://forums.terraria.org/index.php?#1/#2}{\tcbox[on line, boxsep=1pt, colframe=terrariaforumtext, colback=terrariaforumback, left=0pt, right=0pt, top=0pt, bottom=0pt]{\includegraphics[align=c, height=9pt]{figure/terrariaforum.png} \footnotesize\sffamily\textcolor{terrariaforumtext}{#1/#2}}}}
	\newcommand{\tieba}[1]{\href{https://tieba.baidu.com/p/#1}{\tcbox[on line, boxsep=1pt, colframe=blue, colback=white, left=0pt, right=0pt, top=0pt, bottom=0pt]{\includegraphics[align=c, height=9pt]{figure/tieba.png} \footnotesize\sffamily\textcolor{black}{#1}}}}
	\NewDocumentCommand{\bbstr}{O{t}m}{\href{https://www.bbstr.net/#1/#2}{\tcbox[on line, boxsep=1pt, colframe=bbstrtext, colback=bbstrback, left=0pt, right=0pt, top=0pt, bottom=0pt]{\includegraphics[align=c, height=9pt]{figure/bbstr.png} \footnotesize\sffamily\textcolor{bbstrtext}{#1/#2}}}}
	\NewDocumentCommand{\github}{m}{\href{https://github.com/#1}{\tcbox[on line, boxsep=1pt, colframe=black, colback=white, left=0pt, right=0pt, top=0pt, bottom=0pt]{\includegraphics[align=c, height=9pt]{figure/github.pdf} \footnotesize\sffamily #1}}}
	\NewDocumentCommand{\youku}{m}{\href{https://v.youku.com/v_show/id_#1}{\tcbox[on line, boxsep=1pt, colframe=youkudeeppink, colback=white, left=0pt, right=0pt, top=0pt, bottom=0pt]{\includegraphics[align=c, height=9pt]{figure/youku.pdf} \footnotesize\sffamily\textcolor{youkudeeppink}{#1}}}}
\fi

\newcommand{\wiki}[1]{\myhref{https://terraria.wiki.gg/zh/wiki/#1}{#1}}
\newcommand{\wikii}[2]{\myhref{https://terraria.wiki.gg/zh/wiki/#1}{#2}}

\NewDocumentCommand{\vipbox}{mmm}{\href{#3}{\tcbox[on line, sharp corners=all, boxrule=0pt, boxsep=0pt, colback=white, left=0pt, right=0pt, top=0pt, bottom=0pt]{\includegraphics[align=c, height=12pt]{figure/#1}\hspace{1pt}\small\sffamily#2}}\ }
\NewDocumentCommand{\notvip}{m}{{\small\sffamily#1\ }}
% 标准名称

\usepackage{xstring}

\newcommand{\vip}[1]{\IfEqCase{#1}{
	{888}{\vipbox{putianyi888.jpg}{putianyi888}{https://space.bilibili.com/34937101}}
	{adc}{\vipbox{adcakc.jpg}{隐士菌\_AdcAKC}{https://space.bilibili.com/398780730}}
	{cryp}{\vipbox{xxcrypticnightxx.jpg}{xXCrypticNightXx}{https://forums.terraria.org/index.php?members/xxcrypticnightxx.98963}}
	{cyril}{\vipbox{cyril.jpg}{-Cyril-}{https://space.bilibili.com/164033989}}
	{dcf}{\vipbox{dcfhft.jpg}{dcfhft}{https://tieba.baidu.com/home/main/?id=tb.1.cba18462.5MKELc586GMkcKTUO0XLkA}}
	{dice}{\vipbox{dicemanx.jpg}{DicemanX}{https://forums.terraria.org/index.php?members/dicemanx.1706}}
	{drkv}{\vipbox{drkv.jpg}{DRKV}{https://forums.terraria.org/index.php?members/drkv.67603/}}
	{eki}{\vipbox{ekinator.jpg}{ekinator}{https://forums.terraria.org/index.php?members/ekinator.61186}}
	{mappy}{\vipbox{mappygaming.jpg}{Mappygaming}{https://forums.terraria.org/index.php?members/mappygaming.43152}}
	{pro}{\vipbox{programmatic.png}{Programmatic}{https://forums.terraria.org/index.php?members/programmatic.37545}}
	{room}{\vipbox{room.jpg}{ROOM-屠宰场}{https://space.bilibili.com/35610991}}
	{sov}{\vipbox{sovereignvis.jpg}{SovereignVis}{https://forums.terraria.org/index.php?members/sovereignvis.6297}}
	{us}{\vipbox{usdanger.jpg}{us\_danger}{https://tieba.baidu.com/home/main?id=tb.1.b1897b2c.sYKuX9NekGAK7Zq3__6Osg}}
	{yfd}{\vipbox{chenshui.jpg}{混乱沉睡}{https://space.bilibili.com/22871583}}
	{zero}{\vipbox{zerogravitas.jpg}{ZeroGravitas}{https://forums.terraria.org/index.php?members/zerogravitas.96}}
}}

\renewcommand{\figureautorefname}{图}
\renewcommand{\tableautorefname}{表}
\renewcommand{\sectionautorefname}{节}
\renewcommand{\subsectionautorefname}{小节}
\renewcommand{\chapterautorefname}{章}
\renewcommand{\algorithmautorefname}{算法}
\renewcommand{\algorithmcfname}{算法}
\newcommand{\probautorefname}{例题}
\renewcommand\thesubfigure{(\alph{subfigure})}
\newcommand{\subfigureautorefname}{\figureautorefname}
\renewcommand\thesubtable{(\alph{subtable})}
\newcommand{\subtableautorefname}{\tableautorefname}